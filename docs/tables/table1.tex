% Requires: \usepackage{booktabs, pifont, xcolor, array}
% Define symbols (place in preamble)
% \newcommand{\yes}{\textcolor{green!50!black}{\ding{51}}}
% \newcommand{\no}{\textcolor{red!40!black}{\ding{55}}}
% \newcommand{\pmark}{\textcolor{orange!70!black}{$\triangle$\textsuperscript{\dag}}}

\begin{table}[t]
\centering
\begin{adjustbox}{max width=\linewidth}
\begin{tabular}{@{}lcccc@{}}
\toprule
\textbf{Work} & \textbf{Score} & \textbf{Causal} & \textbf{Train-Free} & \textbf{Data-Inv} \\
\midrule
Lynch et al.~(\citeyear{lynch2024eight}) & \no & \no & \no & \pmark \\
Guo et al.~(\citeyear{guo2025mechanistic}) & \no & \pmark & \no & \no \\
Hong et al.~(\citeyear{hong2025intrinsic}) & \no & \pmark & \yes & \no \\
Patil et al.~(\citeyear{patil2024can}) & \no & \no & \yes & \yes \\
Hong et al.~(\citeyear{hong2024dissecting}) & \yes & \yes & \yes & \no \\
\midrule
\textbf{UDS (Ours)} & \yes & \yes & \yes & \yes \\
\bottomrule
\end{tabular}
\end{adjustbox}

\vspace{5pt}
\caption{Comparison of white-box unlearning analysis.
\textbf{Score}: proposes a metric quantifying retained knowledge.
\textbf{Causal}: controlled internal intervention (\pmark{} = causal localization but observational assessment).
\textbf{Train-Free}: no auxiliary training.
\textbf{Data-Inv}: applicable to new forget sets without dataset-specific adaptation (\pmark{} = general approach but requires retraining probes).
}
\label{tab:related-work}
\end{table}
