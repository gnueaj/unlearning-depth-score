% Requires: \usepackage{booktabs, pifont, xcolor, array}
% Define symbols (place in preamble)
% \newcommand{\yes}{\textcolor{green!50!black}{\ding{51}}}
% \newcommand{\no}{\textcolor{red!40!black}{\ding{55}}}
% \newcommand{\pmark}{\textcolor{orange!70!black}{$\triangle$\textsuperscript{\dag}}}

\begin{table}[t]
\centering
\begin{adjustbox}{max width=\linewidth}
\begin{tabular}{@{}lcccc@{}}
\toprule
\textbf{Work}              & \textbf{Train-Free} & \textbf{Causal} & \textbf{Data-Inv} & \textbf{Score} \\
\midrule
\citet{lynch2024eight}     & \no  & \no    & \yes & \no  \\
\citet{guo2025mechanistic} & \no  & \pmark & \no  & \no  \\
\citet{hong2025intrinsic}  & \yes & \pmark & \no  & \no  \\
\citet{patil2024can}       & \yes & \no    & \yes & \no  \\
\citet{hong2024dissecting} & \yes & \yes   & \no  & \yes \\
\midrule
\textbf{UDS (Ours)}        & \yes & \yes   & \yes & \yes \\
\bottomrule
\end{tabular}
\end{adjustbox}
\caption{Comparison of white-box unlearning analysis.
\textbf{Train-Free}: no auxiliary training.
\textbf{Causal}: evaluates knowledge causally (\pmark{} = causal localization but observational assessment).
\textbf{Data-Inv}: directly applicable to new forget sets.
\textbf{Score}: proposes a metric quantifying residual target knowledge.
}
\label{tab:related-work}
\end{table}
