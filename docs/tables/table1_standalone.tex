\documentclass[11pt]{article}
\usepackage[a4paper,margin=2.5cm]{geometry}
\usepackage{times}
\usepackage{booktabs}
\usepackage{pifont}
\usepackage[table]{xcolor}
\usepackage{array}
\usepackage{adjustbox}
\usepackage{natbib}

\newcommand{\yes}{\textcolor{green!50!black}{\ding{51}}}
\newcommand{\no}{\textcolor{red!40!black}{\ding{55}}}
\newcommand{\pmark}{\textcolor{orange!70!black}{$\triangle$\textsuperscript{\dag}}}

\begin{document}
\begin{table}[h]
\centering
\begin{adjustbox}{max width=\linewidth}
\begin{tabular}{@{}lcccc@{}}
\toprule
\textbf{Work} & \textbf{Train-Free} & \textbf{Causal} & \textbf{Data-Inv} & \textbf{Score} \\
\midrule
Lynch et al.~(2024) & \no & \no & \yes & \no \\
Guo et al.~(2025) & \no & \pmark & \no & \no \\
Hong et al.~(2025) & \yes & \pmark & \no & \no \\
Patil et al.~(2024) & \yes & \no & \yes & \no \\
Hong et al.~(2024a) & \yes & \yes & \no & \yes \\
\midrule
\textbf{UDS (Ours)} & \yes & \yes & \yes & \yes \\
\bottomrule
\end{tabular}
\end{adjustbox}

\vspace{5pt}
\caption{Comparison of white-box unlearning analysis.
\textbf{Train-Free}: no auxiliary training.
\textbf{Causal}: evaluates knowledge causally (\pmark{} = causal localization but observational assessment).
\textbf{Data-Inv}: directly applicable to new forget sets.
\textbf{Score}: proposes a metric quantifying residual target knowledge.
}
\label{tab:related-work}
\end{table}
\end{document}
